% Accessible colors from https://www.idpwd.com.au/resources/style-guide/
\definecolor{color0}{RGB}{247,127,0}
\definecolor{color1}{RGB}{81,181,224}
% The original one is too bright for beamers
%\definecolor{color2}{RGB}{206,224,7}
\definecolor{color2}{RGB}{140,190,0}
\definecolor{color3}{RGB}{15,43,91}

\definecolor{vsClass}{RGB}{0,110,150}
\definecolor{vsType}{RGB}{0,0,255}
\definecolor{vsString}{RGB}{215,160,135}
\definecolor{vsComment}{RGB}{10,100,30}

\pgfplotscreateplotcyclelist{mycolors}{
	{fill=color0},
	{fill=color1},
	{fill=color2},
	{fill=color3},
}

\lstset{
  numbers=left,                   % where to put the line-numbers
  numberstyle=\tiny\color{gray},  % the style that is used for the line-numbers
  stepnumber=1,
  numbersep=5pt,                  % how far the line-numbers are from the code
%  backgroundcolor=\color{gray!35},
  showspaces=false,               % show spaces adding particular underscores
  showstringspaces=false,         % underline spaces within strings
  showtabs=false,
  frame=single,                   % adds a frame around the code
  rulecolor=\color{black},        % if not set, the frame-color may be changed
  tabsize=4,                      % sets default tabsize to 4 spaces
  captionpos=b,                   % sets the caption-position to bottom
  breaklines=true,                % sets automatic line breaking
  breakatwhitespace=false,
  language=C,
  keywordstyle=\bfseries\color{vsType},
  commentstyle=\itshape\color{vsComment},
  stringstyle=\color{vsString},
  keywordstyle=[2]{\color{vsClass}},
  escapechar=ß,
  basicstyle=\scriptsize,
  morekeywords={u32, __u32, __be32, __le32,
		u16, __u16, __be16, __le16,
		u8,  __u8,  __be8,  __le8,
		size_t, ssize_t}
}

\lstdefinelanguage[amdgpu]{Assembler}{
	keywords={v_cmp_neq_f32_e32, s_cbranch_execz, s_branch, s_endpgm, image_sample, s_and_saveexec_b64, s_waitcnt,
		s_or_b64, v_mov_b32_e32, v_mul_f32_e32, exp},
	keywords=[2]{v, v0, v1, v2, v3, v4, s, s0, s1, s2, s3, s4, vcc, exec},
	keywords=[3]{vmcnt},
	comment=[l]{;}
}

\lstdefinestyle{glsl}{
	language=C,
	morekeywords={layout, ivec3, out}
}

% https://tex.stackexchange.com/questions/55068/is-there-a-tikz-equivalent-to-the-pstricks-ncbar-command
\tikzset{
	ncbar angle/.initial=90,
	ncbar/.style={
	to path=(\tikztostart)
		-- ($(\tikztostart)!#1!\pgfkeysvalueof{/tikz/ncbar angle}:(\tikztotarget)$)
		-- ($(\tikztotarget)!($(\tikztostart)!#1!\pgfkeysvalueof{/tikz/ncbar angle}:(\tikztotarget)$)!\pgfkeysvalueof{/tikz/ncbar angle}:(\tikztostart)$)
		-- (\tikztotarget)
	},
	ncbar/.default=0.5cm,
}

\tikzset{square left brace/.style={ncbar=0.5cm}}
\tikzset{square right brace/.style={ncbar=-0.5cm}}

% https://tex.stackexchange.com/questions/75836/creating-a-seamless-xor-symbol-as-node
\tikzset{XOR/.style={draw,circle,append after command={
		[shorten >=\pgflinewidth, shorten <=\pgflinewidth,]
		(\tikzlastnode.north) edge (\tikzlastnode.south)
		(\tikzlastnode.east) edge (\tikzlastnode.west)
		}
	}
}

% https://tex.stackexchange.com/questions/13793/beamer-alt-command-like-visible-instead-of-like-only
\makeatletter
% Detect mode. mathpalette is used to detect the used math style
\newcommand<>\AAlt[2]{%
    \begingroup
    \ifmmode
        \expandafter\mathpalette
        \expandafter\math@AAlt
    \else
        \expandafter\make@AAlt
    \fi
    {{#1}{#2}{#3}}%
    \endgroup
}

% Un-brace the second argument (required because \mathpalette reads the three arguments as one
\newcommand\math@AAlt[2]{\math@@AAlt{#1}#2}

% Set the two arguments in boxes. The math style is given by #1. \m@th sets \mathsurround to 0.
\newcommand\math@@AAlt[3]{%
    \setbox\z@ \hbox{$\m@th #1{#2}$}%
    \setbox\@ne\hbox{$\m@th #1{#3}$}%
    \@AAlt
}

% Un-brace the argument
\newcommand\make@AAlt[1]{\make@@AAlt#1}

% Set the two arguments into normal boxes
\newcommand\make@@AAlt[2]{%
    \sbox\z@ {#1}%
    \sbox\@ne{#2}%
    \@AAlt
}

% Place one of the two boxes using \rlap and place a \phantom box with the maximum of the two boxes
\newcommand\@AAlt[1]{%
    \alt#1%
        {\rlap{\usebox0}}%
        {\rlap{\usebox1}}%
    \setbox\tw@\null
    \ht\tw@\ifnum\ht\z@>\ht\@ne\ht\z@\else\ht\@ne\fi
    \dp\tw@\ifnum\dp\z@>\dp\@ne\dp\z@\else\dp\@ne\fi
    \wd\tw@\ifnum\wd\z@>\wd\@ne\wd\z@\else\wd\@ne\fi
    \box\tw@
}

\makeatother

\newcommand<>\Alt[2]{{%
    \sbox0{$\displaystyle #1$}%
    \sbox1{$\displaystyle #2$}%
    \alt#3%
        {\rlap{\usebox0}\vphantom{\usebox1}\hphantom{\ifnum\wd0>\wd1 \usebox0\else\usebox1\fi}}%
        {\rlap{\usebox1}\vphantom{\usebox0}\hphantom{\ifnum\wd0>\wd1 \usebox0\else\usebox1\fi}}%
	}}

\sisetup{per-mode=fraction, exponent-product=\cdot}
\pgfplotsset{width=10cm, height=8cm, grid style = dotted, compat=1.16,
	every tick label/.append style={font=\scriptsize}}

\lstset{
	numbers=left,
	numberstyle=\tiny\color{gray},
	stepnumber=1,
	numbersep=5pt,
	showspaces=false,
	showstringspaces=false,
	showtabs=false,
	frame=single,
	rulecolor=\color{black},
	tabsize=4,
	captionpos=b,
	breaklines=true,
	breakatwhitespace=false,
	language=C,
	commentstyle=\itshape\color{Mahogany},
	stringstyle=\color{BrickRed},
	keywordstyle=\bfseries\color{OliveGreen},
	keywordstyle=[2]{\color{MidnightBlue}},
	keywordstyle=[3]{\color{RoyalPurple}},
	escapechar=ß,
	xleftmargin=8pt,
	xrightmargin=3pt,
	basicstyle=\scriptsize\ttfamily,
	morekeywords={function, in, not}
}

\begin{luacode*}
	function runtexcommand(args)
	local handle = io.popen("../scripts/tex.py " .. args)
	local result = handle:read("*a")
	handle:close()
	-- Replace newlines
	for line in unicode.utf8.gmatch(result, "[^\r\n]+") do
	tex.sprint(line)
	end
	end
\end{luacode*}
\newcommand\runtex[1]{\directlua{runtexcommand([[#1]])}}
