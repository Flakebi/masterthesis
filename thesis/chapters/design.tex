\clearpage
\bigsection{Design}
The LLVM project, their compiler framework and their C compiler clang successfully use \gls{bb} counters for profile-guided optimizations on CPUs. Therefore BB counters suggest themselves to be used for PGO on GPUs.

Where to integrate that instrumentation: In the llpc patcher

How does it work with pal?
How do we bring the information back to the cpu and write them to disk?

Implementation:

How to call it: command line option or environment variable (with shader hash?)

\subsection{Workflow}
\label{sub:workflow}
\Gls{pgo} is a common technique for applications running on CPUs.
Developers are already used to the way PGO is applied.
When using a compiler that builds upon the LLVM framework, the workflow is the following:
\begin{enumerate}
	\item The application is compiled with profiling instrumentation
	\item The application is run $n$ times and produces an output file on each run
	\item The $n$ files are merged into a single file which contains all profiling information
	\item The application is compiled again and optimized using the data from the generated file
\end{enumerate}
As the AMD Vulkan driver is based on LLVM and developers already know this workflow, it suggests itself to use the same workflow for PGO on GPUs.
When using PGO with \texttt{clang}, we use command line arguments to tell LLVM which options we want to use. When invoking the compiler through the Vulkan API, we cannot use command line arguments.
To keep the usage and implementation of options simple, we use environment variables to pass filenames and other options.

\subsection{Optimizations}
The goal of this thesis is to build the instrumentation, which is used in the first step of the PGO workflow. The instrumentation collects information at runtime and is needed to perform optimizations afterwards. Before we can start collecting data, we need to know which information we need. Therefore, it is part of this thesis is to find out which information can enable optimizations and should be collected and also how they can be collected.

The needed data is defined by the optimizations we want to perform in step three. In the following, we list some possible optimizations and the respectively needed profiling data. They are collected from various sources, e.g. PGO optimizations for CPUs~\cite{MicrosoftPgo} and tailored to GPUs, as there are some peculiarities which have to be taken into account. Some described optimizations are specific to the current architecture of GPUs and have no close counterpart on CPUs.

\subsubsection{Improve linearization}
\paragraph{Needed data} Branch probabilities
\paragraph{Description} As described in the introduction, moving seldom taken branches after the return instruction improves the performance.
\paragraph{Background} This optimization is implemented in LLVM~\cite{llvmLinearization}. To avoid generating worse code, it is only applied when profiling data is available or the compiler assigned a very high probability to the hot branch.
\paragraph{Example}\ \\
\begin{minipage}{.47\textwidth}
	\begin{lstlisting}[caption={Linearization --- unoptimized},frame=tlrb,language={[amdgpu]Assembler}]
	; v0 == 0
	v_cmp_neq_f32_e32 vcc, 0, v0
	s_and_saveexec_b64 s[0:1], vcc
	s_cbranch_execz else
	; then-branch
	s_branch endif
else:
	; else-branch
endif:
	s_endpgm
	\end{lstlisting}
\end{minipage}\hfill
\begin{minipage}{.47\textwidth}
	\begin{lstlisting}[caption={Linearization --- optimized},frame=tlrb,language={[amdgpu]Assembler}]
	v_cmp_neq_f32_e32 vcc, 0, v0
	s_and_saveexec_b64 s[0:1], vcc
	s_cbranch_execz else
	; then-branch
endif:
	s_endpgm    ; no jump if the condition is true
else:
	; else-branch
	s_branch endif
	\end{lstlisting}
\end{minipage}

\subsubsection{Hoist Texture Loads from Branches}
\paragraph{Needed data} Branch probabilities
\paragraph{Description} Texture fetches that happen in branches are sometimes moved before the branch by the compiler to be executed unconditionally. This has the effect that while we wait for the texture content to be loaded from memory, the shader can do computational work and does not have to wait. The optimization speeds up the execution if the branch condition for the texture fetch evaluates to true. However, if the condition evaluates to false, it will degrade performance because the GPU unnecessarily fetches memory and occupies bandwidth.
% TODO Cite something?
\paragraph{Background} Some GPU drivers are hoisting texture fetches out of branches, with guessing the probability that a branch is taken.
\paragraph{Example}\ \\
\begin{minipage}{.47\textwidth}
	\begin{lstlisting}[caption={Load Hoisting --- unoptimized},frame=tlrb,language={[amdgpu]Assembler}]
	v_cmp_neq_f32_e32 vcc, 0, v0
	; Set EXEC mask to cmp result
	s_and_saveexec_b64 s[0:1], vcc
	; Skip branch if no lane is active
	s_cbranch_execz endif
	; Load from texture
	image_sample v[0:3], v[5:6], s[0:7], s[12:15] dmask:0xf
	; Wait until texture is loaded
	s_waitcnt vmcnt(0)
endif:
	s_or_b64 exec, exec, s[0:1]
	\end{lstlisting}
\end{minipage}\hfill
\begin{minipage}{.47\textwidth}
	\begin{lstlisting}[caption={Load Hoisting --- optimized},frame=tlrb,language={[amdgpu]Assembler}]
	image_sample v[0:3], v[5:6], s[0:7], s[12:15] dmask:0xf
	v_cmp_neq_f32_e32 vcc, 0, v0
	s_and_saveexec_b64 s[0:1], vcc
	s_cbranch_execz endif
	s_waitcnt vmcnt(0)
endif:
	s_or_b64 exec, exec, s[0:1]
	\end{lstlisting}
\end{minipage}

\subsubsection{Improve register allocation and occupancy}
\paragraph{Needed data} Branch probabilities
\paragraph{Description} The amount of registers that a shader uses can limit the amount of shaders that run on a single compute unit in \gls{smt} fashion. If a single instance of a shader needs many registers, we cannot have multiple instances of them running on the same compute unit because the register memory is not large enough to hold the registers for more instances at the same time.
Therefore, reducing the registers that a shader needs or spilling registers which are rarely needed into memory can improve performance. However, we should not spill registers in the hot path, as this will degrade performance.
Branch probabilities can provide the information, where we can spill registers and where not.
\paragraph{Background} Deciding spilled registers based on profiling information is already done for CPUs~\cite{Bakhvalov2019}. A difference is that on CPUs, the amount of registers is not dynamic but static. So storing variables in memory instead of registers to use fewer registers is never a performance gain.
\paragraph{Example} For example a game has a single shader for most of the objects that get drawn. This shader implements all material properties though not all properties are active for each object.
One of the effects --- e.g. subsurface scattering --- needs a lot of registers but is enabled for few objects only. The high maximum number of used registers forces a low occupancy.
In this case, spilling registers in the subsurface scattering implementation leads to lower register usage, higher occupancy and improves the performance.

\subsubsection{Value Profiling}
\paragraph{Needed data} Variable value distribution
\paragraph{Description} If a variable has a constant value in most executions, we can introduce a branch and apply constant propagation to provide a fast path. This increases the code size, as parts of the code are duplicated, but can speed up the shader for the common case.
\paragraph{Background} The concept of value profiling exists since 1997~\cite{Calder1997}. But there is no detailed profiling for GPUs so far, so this is only done for CPUs. In LLVM, value profiling is used to inline e.g. \texttt{memcpy} calls when the size of the copied memory region is small enough.
\paragraph{Example} A shader may compute textures for multiple material effects and blend them together in the end by using a material texture with weights for every effect.
If an effect is used seldom, its weight is often zero. We should skip the computation of the effect in this case.\ \\
\begin{minipage}{.47\textwidth}
	\begin{lstlisting}[caption={Value Profiling --- unoptimized},frame=tlrb,language={[amdgpu]Assembler}]
	v_mov_b32_e32 v1, 0
	; Lots of computations, result is in v1
	; The weight for the effect is in v0
	v_mul_f32_e32 v1, v1, v0
	\end{lstlisting}
\end{minipage}\hfill
\begin{minipage}{.47\textwidth}
	\begin{lstlisting}[caption={Value Profiling --- optimized},frame=tlrb,language={[amdgpu]Assembler}]
	v_mov_b32_e32 v1, 0
	; Skip if v0 is zero
	v_cmp_eq_f32_e32 vcc, 0, v0
	s_and_saveexec_b64 s[0:1], vcc
	s_cbranch_execz endif
	; Lots of computations
	v_mul_f32_e32 v1, v1, v0
endif:
	s_or_b64 exec, exec, s[0:1]
	\end{lstlisting}
\end{minipage}

\subsubsection{Detect Uniform Computations}
\paragraph{Needed data} Variable value distribution on a SIMD unit
\paragraph{Description} If we do a computation or memory fetch on the same value for every lane on a SIMD unit, it is uniform. Uniform code can be optimized to be executed only once by the scalar unit instead of the vector unit.
\paragraph{Background} This is similar to value profiling but specialized to GPUs. We do not look at isolated executions of a program. Instead, we inspect if values are differing across one parallel execution. Compared to value profiling, we do not need to store used values but only if a value was uniform or not. Using scalar instructions yields great benefits as described in a paper by Chen~\cite{Chen2016}.
\paragraph{Example} If a shader loads an element from a uniform buffer, and we know through instrumentation that the offset into the uniform buffer is usually the same on all SIMD lanes, we can optimize that load.
The compiler can insert a branch which checks if the offset is uniform at runtime. If it is uniform, the scalar unit loads the element from the uniform buffer.
If the offset is not uniform at runtime, we fall back to the original, non-optimized variant.
%\ \\
%\begin{minipage}{.47\textwidth}
%	\begin{lstlisting}[caption={Uniformity --- unoptimized},frame=tlrb,language={[amdgpu]Assembler}]
%	; Use v0 as index
%	buffer_store_dword v1, v0, s[0:3], 0 offen
%	\end{lstlisting}
%\end{minipage}\hfill
%\begin{minipage}{.47\textwidth}
%	\begin{lstlisting}[caption={Uniformity --- optimized},frame=tlrb,language={[amdgpu]Assembler}]
%	v_cmp_eq_f32_e32 vcc, 320, v0
%	s_and_saveexec_b64 s[0:1], vcc
%	s_cbranch_execz else
%	; Use a static offset
%	buffer_store_dword v1, off, s[0:3], 0 offset:320
%	s_branch endif
%else:
%	; Fall back to the original non-uniform case
%	buffer_store_dword v1, v0, s[0:3], 0 offen
%	; ...
%endif:
%	s_or_b64 exec, exec, s[0:1]
%	\end{lstlisting}
%\end{minipage}

\subsubsection{Skip branches}
\paragraph{Needed data} Branch probabilities and distribution on a SIMD unit
\paragraph{Description} If the compiler encounters a non-uniform branch, it inserts code to set the EXEC mask and a jump which skips the branch on the SIMD unit if no lane wants to execute it. If the compiler knows that the branch will be executed (almost) always, it can omit the skip-jump and save an instruction.
\paragraph{Background} This optimization is an example which has no counterpart for CPUs. CPUs do not have the ability to run SIMD instructions only on a part of the lanes. This is a special case of detecting uniform computations where we are interested in the uniformity of the branch-condition.
\paragraph{Example} This optimization saves only a single instruction. An example where this can be useful nonetheless is in a hot loop where optimizing away a single instruction has a measurable impact.

\subsection{Storing Data}
\label{sub:save-design}
On CPUs, LLVM generates code for writing the collected data into a file.
Most of the file writing code is implemented in a static library which gets linked in when compiling a program with PGO instrumentation.
The code creates a file containing all the profile data, consisting of counters, metadata and variable values if value profiling is enabled.
To call the generated code when the application exits, LLVM uses a global destructor.
The destructors are a list of functions that are called when an application quits.

GPUs currently do not support the destructor list. Additionally, shader code is unable to access the file system of the host.
So directly creating and writing result files from the GPU is impossible, we have to use the driver on the CPU for this task.

We want to reuse the file writing code from LLVM but inside the driver instead of inside the shader.
Therefore we link the static library of LLVM into the driver.
Before calling the dump file function, we map the counters in GPU memory to the CPU and we tell LLVM the name of the file we want to write.
The filename is set by an environment variable and it can contain the pipeline id so the data does not get overwritten when saving multiple pipelines.

\subsection{Basic Block Counting}
\label{sub:design_counters}
In LLVM, the counters are used mostly for linearization. To get the accurate information at this stage in the compiler, the \gls{cfg} is instrumented after most optimizations, when it will not be changed significantly anymore.
An earlier instrumentation can not give us the needed information when basic blocks get duplicated because we would count them as the same block. An early instrumentation can also harm following optimizations, e.g. by making it more difficult to merge duplicated code when different counters get incremented.

After the optimizations, the \gls{cfg} changes one more time on GPUs using the \gls{simt} model: The CFG structurization that makes it possible to run shader code on SIMD units.
We can insert the instrumentation counters either before the CFG structurization or afterwards and count whole SIMD units. When comparing both options, we need to take into account, that the whole SIMD unit always executes the same instruction.
For example if only a single lane executes a BB, e.g. an operation that uses aggregated values from the whole unit, we will get a low counter for this block compared to the rest of the blocks. Nevertheless, it does not make sense to deprioritize the block in the linearization step because the whole unit executes the instruction, just with all lanes masked out but one.
This is a strong argument in favor of inserting the BB counters per unit instead of per lane. For this work, we implemented and compare both methods.

After capturing counters, the next task is to use them. As LLVM is already using basic block counters for linearization if they are available, this is a low-hanging fruit.
The amount of registers that a shader needs can limit the occupancy on a compute unit, i.e. the level of multithreading, which is important to hide memory latency.
If the occupancy is too low, we cannot hide memory latency good enough and the GPU will be waiting a significant amount of time.
To improve that situation, we can spill some registers to (scratch) memory and increase the occupancy in that way.
Basic block counters let us decide which parts of the code are executed seldom or not at all, so we can spill registers there.
To estimate the possible performance improvement without fully implementing this optimization, we remove all parts of the shader code that are never executed and measure the impact on used registers and performance.
We cannot use this strategy in a production driver because code that is not executed in a PGO benchmark scene is not necessarily dead code. However, this test can provide us valuable information if such an optimization can be worth the effort.

\subsection{Removing Unused Code}
\label{sub:design_unused}

\subsection{Uniform Variables}
\label{sub:uniformity}
Uniformity plays an important role for the efficiency of GPUs~\cite{Chen2016}. Uniform operations need to be executed only once per SIMD unit instead of 64 times and they can be run in parallel to other instructions that are executed per lane. These possible optimizations make it desirable to find out which variables are uniform and which divergent. LLVM contains a static divergence analysis for this purpose which uses known sources of divergence like shader input variables and marks all possibly divergent variables. All non-marked variables are known to be uniform.

However, this analysis may not find all uniform variables. Some variables may always be uniform at runtime but the compiler does not have enough knowledge about the application logic to prove that it is always the case. Additionally, it may be interesting to know variables which are uniform in \emph{most} cases. If we know variables that are mostly uniform, we can introduce specialized fast paths and fall back to a slower, non-uniform version if necessary.

The first idea to detect uniform branches is simple. At an if-else, we count the executed basic blocks per SIMD unit, so afterwards we know how often the unit ran the if-block and the else-block. From these numbers we can compute how often both blocks are executed in one run.
If both blocks are executed, we know that the condition is not uniform, otherwise either the else-block or the if-block would have been executed, not both. So we can compute how often a condition is uniform or not.
However, this technique does only work if we have an else-block. Without else-block we cannot compute the uniformity of a condition. In addition to that problem, the uniformity of other variables, e.g. memory addresses is also interesting.
For these reasons we decided to implement a generic analysis that finds out how often a variable is uniform.
