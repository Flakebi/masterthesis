\newpage
\vspace*{3.5cm}
\begin{center}
\begin{minipage}{12.5cm}
\section*{Abstract}
% TODO shorter
Today, GPUs are getting used various fields for an increasing number of diverse purposes. They are used in computer games for the graphics, in high-performance computing, even in spreadsheets or databases to offload computations from the CPU.
One of the parts which enable this development are compilers. They are able to compile code from various shader and general purpose languages into code which can be run on a GPU.
Up to know, the GPU compilers have a shortcoming when it comes to optimizations. They have to decide which code gets executed more often than other code and is more important for optimizations.
The knowledge base for this decision and others is only based on heuristics and may not be correct thus giving away potential for optimizations.

On CPUs, it got common to get precise runtime data by profiling and gave that information to the compiler. This technique is called profile-guided optimizations. It is not yet used for GPUs until now.
While there exist many tools that assist with profiling on GPUs, most of them look at the performance only at the level of single kernel or shader executions and supply general metrics like used memory bandwidth, etc.
There are very few tools available which provide insight into what happens inside a shader.

This work focuses on profiling GPU programs to generate the data which will later be needed for optimizations.

%Actual results.
Basic block counting (implemented in LLVM, needed adjustments)
-> unused code statistics
%code size statistics (#bbs)
Analyze register usage (-> histogram: normal, PGO, remove)
Histogram: Gelöschte BBs (viele, wenige, prozentual)

Remove unused code: Switch-vm + 20\%

Condition/Branch uniformity

Uniformity

discussion: PGO instr and use before/after structurize

%\pagenumbering{arabic}\setcounter{page}{1}
%\begin{abstract}
\end{minipage}
\end{center}

%\newpage
%\vspace*{3.5cm}
%\begin{center}
%\begin{minipage}{12.5cm}

%\section*{Abstrakt}

%Abstrakt deutsch
%\end{minipage}
%\end{center}
